\documentclass[12pt]{article}
\usepackage{longtable}
\usepackage{listings}
\lstset{language=C++}
\lstset{breaklines}
\lstset{extendedchars=false}
\usepackage{syntonly}
\usepackage{fancyhdr}
\usepackage{wallpaper}
\usepackage{amssymb}
\usepackage{amsmath}
\usepackage[colorlinks,CJKbookmarks=true,bookmarksnumbered,linkcolor=red,citecolor=red,plainpages=true,pdfstartview=FitH]{hyperref}
\usepackage{ulem}
\usepackage{graphicx}
\usepackage{ifthen}
\usepackage{authblk}
%\usepackage{titlesec}
\usepackage{multirow}
\usepackage{geometry}
\usepackage{bm}
\usepackage{xcolor}
\usepackage{diagbox}
\usepackage{enumerate}

\title{Introduction to Optimization Theory\\Homework Assignment 5}
\author{Chen Zhiyang, 2017011377}
\date{June 2019}

\setlength{\parindent}{0pt}

\begin{document}
\maketitle

\section*{Ex. 9.5}
\textbf{Proof}

Since $\nabla^2f(\bm{x})\le M\bm{I}$, we have $$f(\bm{y})\le f(\bm{x})+(\nabla f(\bm{x}))^T(\bm{y}-\bm{x})+\dfrac{M}{2}\|\bm{y}-\bm{x}\|_2^2.$$

Plug in $\bm{y}=\bm{x}+t\Delta\bm{x}$, we have $$f(\bm{x}+t\Delta\bm{x})\le f(\bm{x})+t(\nabla f(\bm{x}))^T\Delta\bm{x}+\dfrac{Mt^2}{2}\|\Delta\bm{x}\|_2^2.$$

The stopping criterion $f(\bm{x}+\Delta\bm{x})\le f(\bm{x})+\alpha t\nabla f(\bm{x})\Delta\bm{x}$ is satisfied if $$t(\alpha-1)(\nabla f(\bm{x}))^T\Delta\bm{x}\ge\dfrac{Mt^2}{2}\|\Delta\bm{x}\|_2^2,$$ which means $$t\le-2(1-\alpha)\dfrac{\nabla f(\bm{x})^T\Delta\bm{x}}{M\|\Delta\bm{x}\|_2^2}.$$ Since $0<\alpha<0.5$, we have $$0<t\le-\dfrac{\nabla f(\bm{x})^T\Delta\bm{x}}{M\|\Delta\bm{x}\|_2^2}=t_{\max}.$$

From $\beta^s\le \min(t_{\max},1)$, the number of iterations is upper bounded by $s\le\log_{\beta}\min(t_{\max},1)$.

\section*{Ex. 9.10}
\subsection*{(a)}
$$f'(x)=\dfrac{e^{2x}-1}{e^{2x}+1},f''(x)=\dfrac{4e^{2x}}{(e^{2x}+1)^2}.$$

Initially, $x^{(0)}=1$.

First iteration: $$x^{(1)}=x^{(0)}-\dfrac{f'(x^{(0)})}{f''(x^{(0)})}=-0.813.$$

Second iteration: $$x^{(2)}=x^{(1)}-\dfrac{f'(x^{(1)})}{f''(x^{(1)})}=0.409.$$

(The algorithm converges.)

Initially, $x^{(0)}=1.1$.

First iteration: $$x^{(1)}=x^{(0)}-\dfrac{f'(x^{(0)})}{f''(x^{(0)})}=-1.129.$$

Second iteration: $$x^{(2)}=x^{(1)}-\dfrac{f'(x^{(1)})}{f''(x^{(1)})}=1.234.$$

(The algorithm fails to converge.)

\subsection*{(b)}
$$f'(x)=-\dfrac{1}{x}+1,f''(x)=\dfrac{1}{x^2}.$$

Initially, $x^{(0)}=3$.

First iteration: $$x^{(1)}=x^{(0)}-\dfrac{f'(x^{(0)})}{f''(x^{(0)})}=-3.$$

Note that $x^{(1)}\notin \textbf{dom}\ f$.

\section*{Ex. 10.1}
\subsection*{(a)}
\textbf{Proof}

Label non-singularity of the KKT matrix and the four statements as (0), (1), (2), (3) and (4), respectively.

$(0)\Rightarrow(1)$:

Suppose not, \textit{i.e.}, there exists an $\bm{x}$ s.t. $\bm{Ax}=\bm{Px}=\bm{0}, \bm{x}\neq\bm{0}$, and the KKT matrix is non-singular. However, note that $$\left[\begin{matrix} \bm{P} & \bm{A}^T \\ \bm{A} & \bm{0} \end{matrix}\right]\left[\begin{matrix} \bm{x} \\ \bm{0} \end{matrix}\right]=\bm{0},$$ which means the KKT matrix is singular. This is a contradiction.

$(1)\Rightarrow(0)$:

Suppose the KKT matrix is singular, \textit{i.e.}, $$\left[\begin{matrix} \bm{P} & \bm{A}^T \\ \bm{A} & \bm{0} \end{matrix}\right]\left[\begin{matrix} \bm{x} \\ \bm{y} \end{matrix}\right]=\bm{0},\left[\begin{matrix} \bm{x} \\ \bm{y} \end{matrix}\right]\neq\bm{0}.$$ $$\bm{Px}+\bm{A}^T\bm{y}=\bm{0}\Rightarrow\bm{x}^T\bm{Px}+(\bm{Ax})^T\bm{y}=\bm{x}^T\bm{Px}=\bm{0}\Rightarrow\bm{x}=\bm{0},\bm{y}\neq\bm{0}\Rightarrow\bm{A}^T\bm{y}=\bm{0},$$ which contradicts $\text{rank}\,\bm{A}=p.$

$(1)\Rightarrow(2)$:

Suppose $\bm{Ax}=\bm{x}^T\bm{Px}=\bm{0},\bm{x}\neq\bm{0}$. Since $\bm{P}\ge 0$, we have $\bm{Px}=\bm{0}$, which means $\bm{x}\in\mathcal{N}(\bm{P})\cap\mathcal{N}(\bm{A})$. Contradiction.

$(2)\Rightarrow(4)$:

Let $\bm{Q}=\bm{I}$, obviously we have $\bm{P}+\bm{A}^T\bm{A}>0$.

$(4)\Rightarrow(1)$:

Suppose $\bm{x}\in\mathcal{N}(\bm{P})\cap\mathcal{N}(\bm{A}),\bm{x}\neq\bm{0}$, we have $$\bm{Px}+\bm{Ax}=\bm{0}\Rightarrow \bm{x}^T(\bm{P}+\bm{A}^T\bm{QA})\bm{x}=\bm{0},$$ which is a contradiction.

$(2)\Leftrightarrow(3)$:

If $\bm{Ax}=\bm{0},\bm{x}\neq\bm{0},\bm{x}^T\bm{Px}>0$, since $\bm{x}\in\mathcal{R}(\bm{F})$, $\bm{x}=\bm{Fy}$ for some $\bm{y}\neq\bm{0}$. We have $$\bm{x}^T\bm{Px}=\bm{y}^T(\bm{F}^T\bm{PF})\bm{y}>0\Leftrightarrow\bm{F}^T\bm{PF}>0.$$

The above reductions form a strongly-connected directed graph, \textit{i.e.}, the statements are equivalent.

\section*{Ex. 10.9}
\subsection*{(a)}
It is obvious that $f$ is a convex function. Suppose the dual optimal cost is $p^*$. If $p^*$ is not a feasible cost, either $$\lim_{\bm{x}\rightarrow\infty}f(\bm{x})=p^*,$$ or $$\lim_{\bm{x}\rightarrow\bm{x}^*}f(\bm{x})=p^*,$$ where $\bm{x}^*$ has zero components.

Note that $\lim_{x\rightarrow\infty}x\log x=\infty$, the first situation is impossible.

As for the second situation, let $\Delta\bm{x}=\bm{x}-\bm{x}^*$, $g(t)=\sum (x_i^*+t\Delta x_i)\log(x_i^*+t\Delta x_i), t>0.$ We have $$g'(t)=\sum\Delta x_i(1+\log(x_i^*+t\Delta x_i)).$$ If $x_i^*=0$, $\Delta x_i>0$, which means $\lim_{t\rightarrow 0}g(t)=-\infty$. This is also impossible.

\section*{Ex. 11.4}
Let $\phi$ be the barrier function of \textbf{Ex. 11.1}. We have $$\nabla^2(tf_0+\phi)=\nabla^2(tf_0+\tilde{\phi})+\dfrac{2}{R^2-\|\bm{x}\|_2^2}\bm{I}+\dfrac{4}{(R^2-\|\bm{x}\|_2^2)^2}\bm{x}\bm{x}^T\ge\dfrac{2}{R^2}\bm{I}.$$

We can take $m=\frac{2}{R^2}.$

\end{document}
